\documentclass[12pt]{scrartcl}
\title{Gravity}
\author{I. Saltini}
\date{\today}
\setlength{\parindent}{0pt}

\usepackage{polyglossia}
\setmainlanguage[variant=british]{english}

\usepackage{amssymb,amsmath,amsthm}
\usepackage{booktabs,multirow}
\usepackage{graphicx}
\usepackage{enumerate,enumitem}

\usepackage{caption}
\usepackage{subcaption}
\usepackage[hidelinks,pdfusetitle]{hyperref}
\usepackage[figure]{hypcap}
\usepackage{float}

\usepackage{cool}
\Style{DSymb={\mathrm{d}},IntegrateDifferentialDSymb={\mathrm{d}}}

\usepackage[math-style=ISO,bold-style=ISO]{unicode-math}
\defaultfontfeatures{Ligatures=TeX,ExternalLocation=fonts/,Extension=.otf}
\setmathfont{LibertinusMath}
\setmathfont[range={\mathcal,\mathbfcal},StylisticSet=1]{LibertinusMath}
\defaultfontfeatures{
  Ligatures=TeX,
  ExternalLocation=fonts/,
  Extension=.otf
}
\setmainfont[
  UprightFont=*R,
  ItalicFont=*I,
  BoldFont=*B,
  BoldItalicFont=*BI
]{LibertinusSerif}
\setsansfont[
  UprightFont=*R,
  ItalicFont=*I,
  BoldFont=*B
]{LibertinusSans}
\setmonofont[
  UprightFont=*R
]{LibertinusMono}

\input{headers/si}
%vectors
\NewDocumentCommand\vv{m}{\symbf{#1}}
\NewDocumentCommand\vs{m}{\symbf{u}_{#1}}
\NewDocumentCommand\norm{m}{\|#1\|}

%matrices
\NewDocumentCommand\idn{}{\mathbb{1}}
\NewDocumentCommand\inv{m}{{#1}^{-1}}
\NewDocumentCommand\tr{m}{{#1}^t}
\NewDocumentCommand\dt{m}{\left|#1\right|}

%bracketing
\NewDocumentCommand\rb{m}{\left(#1\right)}

%groups
\NewDocumentCommand\euc{m}{\mathrm{E}\!\left(#1\right)}
\NewDocumentCommand\seuc{m}{\mathrm{SE}\!\left(#1\right)}
\NewDocumentCommand\transl{m}{\mathrm{T}\!\left(#1\right)}
\NewDocumentCommand\gal{m}{\mathrm{Gal}\!\left(#1\right)}
\NewDocumentCommand\galb{m}{\widetilde{\mathrm{B}}\!\left(#1\right)}

% \input{headers/bibliography}
% \input{headers/drawings}
% \input{headers/fancy}

\begin{document}
\maketitle

\section{Introduction to Classical Mechanics}

In Classical Mechanis (CM) time and space are treated as two distinct concepts.
Events are identified by their position in some frame of reference, represented
by a vector \(\vv{r}\), and the time at which they occur, represented by a
scalar \(t\).

The simplest formulation of CM can be summarised in three laws and is based on
the concept of \textbf{force}. A force \(\vv{F}\) is a vector quantity
representing an interaction between two bodies. This formulation is (unduly)
credited to Newton because it was first published as a unified theory in the
\emph{Principia}.

\begin{enumerate}[label=\textbf{N\arabic*},start=0]
  \item \label{law::N0} \textbf{Principle of Superposition:} the net force
  on a body is the vector sum of the individual forces acting on it.
  \item \label{law::N1} \textbf{Law of Inertia:} an inertial frame of reference
  is one where, in the absence of external interactions, bodies are in uniform
  linear motion.
  \item \label{law::N2} \textbf{Law of Acceleration:} in an inertial frame of
  reference the acceleration of a body is directly proportional to the net force
  acting on it.
  \item \label{law::N3} \textbf{Law of Reaction:} when a body exerts a force on
  another it is subject to a force equal in magnitude and opposite in direction
  produced by the second body.
\end{enumerate}

Law \ref{law::N2} can be used to introduce the concept of \textbf{inertial mass}
\(m\), which is taken as the proportionality constant between force and
acceleration. The law can then be stated with the following equation.

\begin{equation}\label{eqn:N2}
    \vv{F} = m \vv{\ddot{r}}\tag{N2}
\end{equation}

\section{Galilean gravity}

The simplest theory of gravity in our possession is due to Galileo

\end{document}
