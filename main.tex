\documentclass[12pt]{scrartcl}
\title{Gravity}
\author{I. Saltini}
\date{}
\setlength{\parindent}{0pt}

\usepackage{polyglossia}
\setmainlanguage[variant=british]{english}

\usepackage{amssymb,amsmath,amsthm}
\usepackage{booktabs,multirow}
\usepackage{graphicx}
\usepackage{enumerate,enumitem}

\usepackage{caption}
\usepackage{subcaption}
\usepackage[hidelinks,pdfusetitle]{hyperref}
\usepackage[figure]{hypcap}
\usepackage{float}

\usepackage{cool}
\Style{DSymb={\mathrm{d}},IntegrateDifferentialDSymb={\mathrm{d}}}

\usepackage[math-style=ISO,bold-style=ISO]{unicode-math}
\defaultfontfeatures{Ligatures=TeX,ExternalLocation=fonts/,Extension=.otf}
\setmathfont{LibertinusMath}
\setmathfont[range={\mathcal,\mathbfcal},StylisticSet=1]{LibertinusMath}
\defaultfontfeatures{
  Ligatures=TeX,
  ExternalLocation=fonts/,
  Extension=.otf
}
\setmainfont[
  UprightFont=*R,
  ItalicFont=*I,
  BoldFont=*B,
  BoldItalicFont=*BI
]{LibertinusSerif}
\setsansfont[
  UprightFont=*R,
  ItalicFont=*I,
  BoldFont=*B
]{LibertinusSans}
\setmonofont[
  UprightFont=*R
]{LibertinusMono}

\input{headers/si}
%vectors
\NewDocumentCommand\vv{m}{\symbf{#1}}
\NewDocumentCommand\vs{m}{\symbf{u}_{#1}}
\NewDocumentCommand\norm{m}{\|#1\|}

%matrices
\NewDocumentCommand\idn{}{\mathbb{1}}
\NewDocumentCommand\inv{m}{{#1}^{-1}}
\NewDocumentCommand\tr{m}{{#1}^t}
\NewDocumentCommand\dt{m}{\left|#1\right|}

%bracketing
\NewDocumentCommand\rb{m}{\left(#1\right)}

%groups
\NewDocumentCommand\euc{m}{\mathrm{E}\!\left(#1\right)}
\NewDocumentCommand\seuc{m}{\mathrm{SE}\!\left(#1\right)}
\NewDocumentCommand\transl{m}{\mathrm{T}\!\left(#1\right)}
\NewDocumentCommand\gal{m}{\mathrm{Gal}\!\left(#1\right)}
\NewDocumentCommand\galb{m}{\widetilde{\mathrm{B}}\!\left(#1\right)}

% \input{headers/bibliography}
% \input{headers/drawings}
% \input{headers/fancy}

\begin{document}
\maketitle

\section{Introduction to Classical Mechanics}

In Classical Mechanis (CM) time and space are treated as two distinct concepts.
Events are identified by their position in some frame of reference, represented by a vector \(\vv{r}\), and the time at which they occur, represented by a scalar \(t\).

The simplest formulation of CM is based on the concept of \textbf{force}.
A force \(\vv{F}\) is a vector quantity representing an interaction between two bodies.

\begin{enumerate}[label=\textbf{N\arabic*},start=0]
  \item \label{law::N0} \textbf{Principle of Superposition:} the net force on a body is the vector sum of the individual forces acting on it.
  \item \label{law::N1} \textbf{Law of Inertia:} an inertial frame of reference is one where, in the absence of a net force, bodies are in uniform linear motion.
  \item \label{law::N2} \textbf{Law of Acceleration:} in an inertial frame of reference the acceleration of a body is directly proportional to the net force acting on it.
  \item \label{law::N3} \textbf{Law of Reaction:} a body exerting a force on another is itself subject to a force equal in magnitude and opposite in direction produced by the second body.
\end{enumerate}

This formulation of mechanics is (unduly) credited to Newton because it was first published as a unified theory in the \emph{Principia}, despite \ref{law::N1} being already known to Galileo and \ref{law::N3} being a consequence of the Cartesian principle of conservation of momentum.
\ref{law::N0} was omitted by Newton and most sources refer to this formulation of mechanics as Newton’s \emph{three} laws as a result.

Law \ref{law::N2} can be used to introduce the concept of \textbf{inertial mass} \(m\), which is taken as the proportionality constant between force and acceleration.
The law can then be stated as

\begin{equation}\label{eqn:N2}
    \vv{F} = m \, \ddot{\vv{r}}\tag{N2}
\end{equation}

\section{Galilean gravity}

The simplest theory of gravity in our possession is due to Galileo and simply states that acceleration due to gravity is constant, regardless of the size, velocity or position of a body.
In hindsight, Galilean gravity can be thought of as gravity in the limit where Earth is and indefinitely extended plane with a uniform mass distribution.
%
\[\ddot{\vv{r}} = \vv{g}\]
%
This theory, although excessively simple, lays the foundation of a fundamental principle.

\begin{enumerate}[label=\textbf{EP\textsubscript{w}}]
  \item \label{law::EPw} \textbf{(Weak) Equivalence Principle:} all bodies subject to a gravitational field will undergo the same acceleration if they are located at the same coordinates.
\end{enumerate}

This principle is backed up by all experimental evidence to date.

\section{Hookean gravity}

Generally credited to Newton, most of the fundamental work on this theory was actually due to Hooke.
The basic principle of the theory is that every body exerts a gravitational force on every other body.
This force pulls the objects towards each other and its intensity is:
\begin{itemize}
  \item directly proportional to the \textbf{active gravitational mass} \(\widetilde{m}^{\text{(a)}}\) of the body exerting the force,
  \item directly proportional to the \textbf{passive gravitational mass} \(\widetilde{m}^{\text{(p)}}\) of the body subject to the force,
  \item inversely proportional to the square of their distance.
\end{itemize}

This can be expressed as the following formula
%
\[F_{1 \to 2} = G \, \frac{\widetilde{m}_1^{\text{(a)}}
\widetilde{m}_2^{\text{(p)}}}{\norm{\vv{r}_{1 \to 2}}^2}\]
%
where \(F_{1 \to 2}\) is the intensity of the force exerted by body 1 on body 2 and \(\vv{r}_{1 \to 2} = \vv{r}_{2} - \vv{r}_1\) is the relative position of body 2 with respect to body 1.
\(G\) is a universal constant know as the \textbf{gravitational constant} or \textbf{Cavendish constant}.

By applying \ref{law::N3} we require that \(F_{1 \to 2} = F_{2 \to 1}\).
This can be manipulated to obtain that the ratio between active and passive gravitational mass must be constant for all bodies.
%
\[\frac{\widetilde{m}_1^{\text{(a)}}}{\widetilde{m}_1^{\text{(p)}}} =
\frac{\widetilde{m}_2^{\text{(a)}}}{\widetilde{m}_2^{\text{(p)}}}\]
%
Therefore we can take this constant to be unity and drop the distinction between the two types of gravitational mass.
Our law then becomes
%
\[F_{1 \to 2} = G \, \frac{\widetilde{m}_1\widetilde{m}_2}
{\norm{\vv{r}_{1 \to 2}}^2}\]
%
We follow up by applying \ref{law::N2} to the second body and finding its
acceleration \(\ddot{r}_2\).
%
\[\ddot{r}_2 = G \, \frac{\widetilde{m}_1}{R^2} \frac{\widetilde{m}_2}{m_2}\]
%
By virtue of the \ref{law::EPw} this acceleration must not depend upon any properties of the body itself, the ratio between gravitational and inertial mass must therefore be a constant.
This constant can also be taken to be unity, removing any and all practical distinctions between the various types of mass.

We thus obtain the \textbf{Law of Universal Gravitation} in its most commonly used forms.

\begin{equation}\label{eqn:UG}
    F_{1 \to 2} = G \, \frac{m_1 m_2}{\norm{\vv{r}_{1 \to 2}}^2}
    \qquad
    \vv{F}_{1 \to 2} = - G \, \frac{m_1 m_2}{\norm{\vv{r}_{1 \to 2}}^3}  \, \vv{r}_{1 \to 2}
    \tag{UG}
\end{equation}

\section{Galilean relativity}

The concept of Galilean relativity can be formulated as:

\begin{enumerate}[label=\textbf{SR\textsubscript{w}}]
  \item \label{law::SRw} \textbf{(Weak) Special Principle of Relativity:} the laws of motion are the same in all frames of reference in uniform linear motion with respect to one another.
\end{enumerate}

An important remark is that such wording of the principle only refers to the laws of \emph{motion}, and therefore to acceleration.
It does not ensure that \emph{all} laws of physics are the same.

\end{document}
